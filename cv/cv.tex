\documentclass[letterpaper]{article}

\usepackage{hyperref}
\usepackage{geometry}

% Comment the following lines to use the default Computer Modern font
% instead of the Palatino font provided by the mathpazo package.
% Remove the 'osf' bit if you don't like the old style figures.
\usepackage[T1]{fontenc}
\usepackage[sc,osf]{mathpazo}

% Set your name here
\def\name{Zheguang Samuel Zhao}

% Replace this with a link to your CV if you like, or set it empty
% (as in \def\footerlink{}) to remove the link in the footer:
%\def\footerlink{http://jblevins.org/projects/cv-template/}

% The following metadata will show up in the PDF properties
\hypersetup{
  colorlinks = true,
  urlcolor = black,
  pdfauthor = {\name},
  pdfkeywords = {computer science, cryptograph, systems, data science},
  pdftitle = {\name: Curriculum Vitae},
  pdfsubject = {Curriculum Vitae},
  pdfpagemode = UseNone
}

\geometry{
  body={6.5in, 8.5in},
  left=1.0in,
  top=1.25in
}

% Customize page headers
\pagestyle{myheadings}
\markright{\name}
\thispagestyle{empty}

% Custom section fonts
\usepackage{sectsty}
\sectionfont{\rmfamily\mdseries\Large}
\subsectionfont{\rmfamily\mdseries\itshape\large}

% Other possible font commands include:
% \ttfamily for teletype,
% \sffamily for sans serif,
% \bfseries for bold,
% \scshape for small caps,
% \normalsize, \large, \Large, \LARGE sizes.

% Don't indent paragraphs.
\setlength\parindent{0em}

% Make lists without bullets
\renewenvironment{itemize}{
  \begin{list}{}{
    \setlength{\leftmargin}{1.5em}
  }
}{
  \end{list}
}

\begin{document}

% Place name at left
{\huge \name}

% Alternatively, print name centered and bold:
%\centerline{\huge \bf \name}

\vspace{0.25in}

\begin{minipage}{0.45\linewidth}
  Brown University \\
  Department of Computer Science \\
  115 Waterman St \\
  Providence, RI 02912 \\
  United States of America
\end{minipage}
\begin{minipage}{0.45\linewidth}
  \begin{tabular}{ll}
    %Phone: & (608) 630-1677 \\
    Email: & \href{mailto:zheguang.zhao@gmail.com}{\tt zheguang.zhao@gmail.com} \\
    Homepage: & \href{https://zheguang.github.io}{\tt zheguang.github.io} \\
    LinkedIn: & \href{https://www.linkedin.com/in/samuelzhao}{\tt www.linkedin.com/in/samuelzhao} \\
    Github: & \href{https://github.com/zheguang}{\tt github.com/zheguang} \\
    Google Scholar: & \href{https://goo.gl/DR8pSa}{\tt goo.gl/DR8pSa} \\
  \end{tabular}
\end{minipage}

\section*{Education}

\begin{itemize}
  \item Ph.D. Candidate in Computer Science, Brown University, expected 2019.\\
  Advisor: Prof. Stan Zdonik, Prof. Seny Kamara
  \item M.S. in Computer Science, Brown University, 2016.\\
  Advisor: Prof. Stan Zdonik
  \item B.S. in Computer Science, University of Wisconsin at Madison, 2012.\\
  Advisor: Prof. Jignesh Patel
\end{itemize}

\section*{Experiences}

\begin{itemize}
\item Sifr Systems, RI, Database Scientist, 2018 -- present.
	\begin{itemize}
		\item Develop provably-secure end-to-end encrypted big data systems including PostgreSQL and 
		Apache Spark
	\end{itemize}
\item Brown University, RI
	\begin{itemize}
		\item Research Assistant, 2014 -- present.
		\item Teaching Assistant, 2015, 2018
	\end{itemize}
\item Microsoft AI \& Research, WA, Research Intern, 2017.
	\begin{itemize}
		\item Research on constraint learning for automatic puzzle solving AI
	\end{itemize}
\item Intel Labs, CA, Research Intern, 2015.
	\begin{itemize}
		\item Research on the efficiency of machine learning algorithms in Apache Spark
		\item Research on in-memory transactional database VoltDB using non-volatile memory
	\end{itemize}
\item Hadapt (Acquired by Teradata), MA, Software Engineer, 2013 -- 2014.
	\begin{itemize}
		\item Develop the enterprise SQL-on-Hadoop system including query execution, storage engine, high availability and analytics toolkit.  Use Agile methodology and continuous integration.
	\end{itemize}
\item Kosmix (Acquired by @WalmartLabs), CA, Software Engineer Intern, 2012.
	\begin{itemize}
		\item Develop an in-memory distributed queue system for the in-house distributed stream processing system in support for data analytics and machine learning
	\end{itemize}
\item Great Lakes Bioenergy Research Center, WI, Software Engineer Intern, 2010 -- 2012.
	\begin{itemize}
		\item Develop biological data management system using .NET and Oracle database
	\end{itemize}
\end{itemize}


\section*{Honors}
\begin{itemize}
\item Eta Kappa Nu
\item Upsilon Pi Epsilon
\item Golden Key International Honour Society
\end{itemize}

\section*{Open-source Projects}
\begin{itemize}
\item Searchable encryption for mobile messaging in Signal \\ \textit{https://github.com/encryptedsystems/Searchable-Signal-Android}
\item Macau: statistical hypothesis testing based on resampling \\ \textit{https://github.com/zheguang/macau}
\item Machine learning algorithms in Spark \\ \textit{https://github.com/zheguang/spark-study/tree/master/study/src/main/scala/edu/brown/cs/sparkstudy}
\item Consistency control for machine learning algorithms \\ \textit{https://github.com/zheguang/babel}
\item R-tree in Rust \\ \textit{https://github.com/zheguang/rtree}
\item Spark performance analysis tool \\ \textit{https://github.com/zheguang/spark-perftool}
\item VoltDB on non-volatile memory \\ \textit{https://github.com/zheguang/voltdb}
\end{itemize}


\section*{Research}

%I am interested in the theories and designs of big data systems that are intelligent and safe.  
%My current study focuses on efficient encrypted SQL for provable security.  
%Relational databases play fundamental roles in supporting today?s cloud computing and Big Data ecosystems.  However, there has been rising public concerns about privacy and security of the data collected and analyzed in the cloud, most notably due to several incidents of sensitive data being compromised at major companies such as Yahoo and Equifax. 
%The main technical challenges of encrypting such data are both in theory and practice. We need to have encryption schemes that are not only more efficient than general-purpose primitives like fully-homomorphic encryption or oblivious RAM, but also have less leakage than the previously known schemes based on property-preserving encryption.  To co-design the SQL database around such encryption schemes requires re-thinking about the query processing, translation, optimization and data storage to achieve optimal, practical performance. 
%In the past, I have also dabbled in constraint learning for puzzle-solving AI, false-discovery control in data science, approximate data structures for visualization, database design on hybrid memory, consistency control for stochastic machine learning algorithms, and searchable encryption on mobile text messaging.

I am interested in the theories and designs of big data systems that are intelligent and safe. My research spans a broad area covering cryptography, data science/machine learning, and big data systems.  

In this spirit I have dabbled in:
\begin{itemize}
	\item Constraint learning for puzzle-solving AI 
		\begin{itemize} 
			\item Can kids teach an AI system to solve their favorite games like Rubik's cube or Sudoku, or even a new game they invent?  This project explores the architecture for such a general-purpose AI puzzle-solving system, from natural language interface, programming by demonstration, knowledge representation, and finally learning the optimal winning strategy of the game.  
		\end{itemize}
	\item False discovery control in data science
		\begin{itemize}
			\item Recommendation engines and human analysts are not very capable of distinguishing true relationships from random noises that are inherent in the data.   What is the way to automatically detect and control the risk of false insights?  We build the first system to systematically guide the data exploration process away from noises.
		\end{itemize}
	\item Approximate data structures for visualization
		\begin{itemize}
			\item How do we visualize a distribution on a cloud-scale dataset within interactive time?  This project investigates how to augment the B-tree index with information to help approximate the underlying data distribution and refine the answer progressively.
		\end{itemize}
	\item Data system design on hybrid memory
		\begin{itemize}
			\item This project studies the design of data systems if the memory hierarchy consists of both a volatile and fast component, and a non-volatile but slightly slower component.  
		\end{itemize}
	\item Consistency control for stochastic machine learning algorithms
		\begin{itemize}
			\item Many machine learning algorithms are shown to converge faster when models are updated stochastically.  Stochasticity leads to parallelism.  This leads to a fundamental question about to what degree of stochasticity can trade off model consistency for shorter training time.  This project studies the consistency levels for machine learning algorithms.
		\end{itemize}
	\item Searchable encryption on mobile text messaging
		\begin{itemize}
			\item We bring the world's first encrypted search to the secure mobile messaging app, Signal, which is widely used by government agencies, journalists, activists and people who are concerned about security and privacy.
		\end{itemize}
\end{itemize}

\section*{Articles}

\begin{itemize}

\item \textit{Behavior of Large Random Graph.}\\
  Z. Zhao, supervised by Prof. Paul Dupius, \\
  Randomized Algorithms for Counting, Integration and Optimization, Brown University, April 2017.

\item \textit{Investigating the Effect of the Multiple Comparisons Problem in Visual Analysis.} \\
  E. Zgraggen, Z. Zhao, R. Zeleznik, and T. Kraska, \\
  CHI, April 2018.

\item \textit{Signal Search.} \\
  J. Engelman, S. Kamara, T. Moataz and S. Zhao, \\
  Software release: \href{http://github.com/encryptedsystems/Searchable-Signal-Android}{\tt http://github.com/encryptedsystems/Searchable-Signal-Android}. \\
  Press release: \href{http://esl.cs.brown.edu/blog/signal}{\tt http://esl.cs.brown.edu/blog/signal}, April 2017.

\item \textit{Controlling False Discoveries During Interactive Data Exploration.} \\
  Z. Zhao, L. De Stefani, E. Zgraggen, C. Binnig, E. Upfal and T. Kraska, \\
  SIGMOD, May 2017.

\item \textit{Safe Visual Data Exploration.} \\
  Z. Zhao, E. Zgraggen, L. De Stefani, C. Binnig, E. Upfal and T. Kraska, \\
  SIGMOD Demo, May 2017.

\item \textit{Bridging the Gap between HPC and Big Data frameworks.} \\
  M. Anderson, S. Smith, N. Sundaram, M. Capota, Z. Zhao, S. Dulloor, N. Satish and T. Willke, \\
  VLDB, 2017.

\item \textit{Towards Sustainable Insights.} \\
  C. Binnig, L. De Stefani, T. Kraska, E. Upfal, E. Zgraggen and Z. Zhao, \\
  CIDR, January 2017.

\item \textit{Towards a Benchmark for Interactive Data Exploration.} \\
  P. Eichmann, E. Zgraggen, Z. Zhao, C. Binnig, T. Kraska. \\
   IEEE Data Engineering Bulletin, 2016.

\item \textit{Larger-than-memory Data Management on Modern Storage Hardware for In-memory OLTP Database Systems.} \\
  L. Ma, J. Arulraj, S. Zhao, A. Pavlo, S. Dulloor, M. Giardino, J. Parkhurst, J. Gardner, K. Doshi and S. Zdonik, \\
  SIGMOD DaMoN, June 2016.

\item \textit{VisTrees: Fast Indexes for Interactive Data Exploration.} \\
  M. El-Hindi, Z. Zhao, C. Binnig and T. Kraska, \\
  SIGMOD HILDA, June 2016.

\item \textit{Data Tiering in Heterogeneous Memory Systems.} \\
  S. Dulloor, A. Roy, Z. Zhao, N. Sundaram, N. Satish, R. Sankaran, J. Jackson and K. Schwan, \\
  EuroSys, April 2016.

\end{itemize}

\section*{Selected Coursework}
\begin{itemize}
\item \textit{Abstract Algebra}, Prof. Rich Schwartz
\item \textit{Calculus}, Prof. Donald Passman, Gheorghe Craciun
\item \textit{Randomized Algorithms for Counting, Integration and Optimization}, Prof. Paul G. Dupuis
\item \textit{Cryptography}, Prof. Seny Kamara, Joseph Silverman
\item \textit{Probability}, Prof. Erik Sudderth, Samuel S. Watson 
\item \textit{Computational Linguistics}, Prof. Eugene Charniak
\item \textit{Computer Architecture}, Prof. Sherief Reda, Mark D. Hill
\item \textit{Distributed Computing through Combinatorial Topology}, Prof. Maurice Herlihy
\item \textit{Database Management}, Prof. Stan Zdonik, Jignesh Patel, Chris R\'e
\item \textit{Microprocessor Synchronization}, Prof. Maurice Herlihy
\item \textit{Algorithms and Data Structures}, Prof. Eric Vigoda, Ben Liblit
\item \textit{Operating Systems}, Prof. Michael Swift
\item \textit{Computer Networks}, Prof. Aditya Akella
\item \textit{Physics}, Prof. Peter Timbie, Daniel Chung, Ellen Zweibel
\end{itemize}


\section*{Reference}
\begin{itemize}

\item Prof. Stanley Zdonik, Professor at Brown University, \textit{sbz@cs.brown.edu}
\item Prof. Seny Kamara, Professor at Brown University, \textit{seny@cs.brown.edu}
\item Dr. Emanuel Zgraggen, Postdoctoral associate at MIT, \textit{emanuel.zgraggen@gmail.com}
\item Dr. Subramanya Dulloor, Intel Labs, \textit{dulloor@gmail.com}
\item Dr. Wang Lam, WalmartLabs, \textit{wlam@cs.stanford.edu}

\end{itemize}

\bigskip

\end{document}
